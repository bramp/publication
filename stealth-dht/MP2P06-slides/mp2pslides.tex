\documentclass{beamer}

\mode<presentation> {
  %\usetheme{Antibes}
  %\usetheme{Bergen}
  %\usetheme{Berkeley}
  %\usetheme{Berlin}
  %\usetheme{Boadilla}
  %\usetheme{boxes}
  %\usetheme{Copenhagen}
  %\usetheme{Darmstadt}
  %\usetheme{default}
  %\usetheme{Dresden}
  %\usetheme{Frankfurt}
  %\usetheme{Goettingen}
  %\usetheme{Hannover}
  %\usetheme{Ilmenau}
  %\usetheme{JuanLesPins}
  %\usetheme{Luebeck}
  %\usetheme{Madrid}
  %\usetheme{Malmoe}
  %\usetheme{Marburg}
  %\usetheme{Montpellier}
  %\usetheme{PaloAlto}
  %\usetheme{Pittsburgh}
  %\usetheme{Rochester}
  %\usetheme{Singapore}
  %\usetheme{Szeged}
  \usetheme{Warsaw}
  \setbeamercovered{transparent}
}

\usepackage[english]{babel}

\usepackage[latin1]{inputenc}

\usepackage{times}
\usepackage[T1]{fontenc}


% adds underlines/strikethroughs
\usepackage{ulem}

\title[Stealth DHT] % (optional, use only with long paper titles)
{Performance Analysis of Stealth Distributed Hash Table with Mobile
Nodes}

%\subtitle
%{Presentation Subtitle} % (optional)

%\author[] % (optional, use only with lots of authors)
%{F.~Author\inst{1} \and S.~Another\inst{2}}
%{Andrew Brampton\\a.brampton@lancs.ac.uk}
% - Use the \inst{?} command only if the authors have different
%   affiliation.

\institute[Lancaster University, UK] % (optional, but mostly needed)
{
%  \inst{1}%
  Andy MacQuire, \textbf{Andrew Brampton}, Idris A. Rai and Laurent Mathy\\
  \{macquire,brampton,rai,laurent\}@comp.lancs.ac.uk\\
  Computing Department\\
  Lancaster University, UK
}

\date[18th March 2006] % (optional)
{Friday 18th March / Mobile Peer-to-Peer 2006}

\subject{Stealth Distributed Hash Table}
% This is only inserted into the PDF information catalog. Can be left
% out.

 \pgfdeclareimage[height=1.1cm]{university-logo}{lancslogo}
 \logo{ \pgfuseimage{university-logo} }

% If you wish to uncover everything in a step-wise fashion, uncomment
% the following command:

%\beamerdefaultoverlayspecification{<+->}

% suppresses all navigation symbols
\setbeamertemplate{navigation symbols}{}

\begin{document}

\begin{frame}
  \titlepage
\end{frame}

%\section{Table Of Contents}

\begin{frame}
  \frametitle{Outline}
  \tableofcontents
  % You might wish to add the option [pausesections]
\end{frame}


% Since this a solution template for a generic talk, very little can
% be said about how it should be structured. However, the talk length
% of between 15min and 45min and the theme suggest that you stick to
% the following rules:

% - Exactly two or three sections (other than the summary).
% - At *most* three subsections per section.
% - Talk about 30s to 2min per frame. So there should be between about
%   15 and 30 frames, all told.

\section{Introduction}

%\subsection[DHTs - Overview]{DHTs - Overview}
%
%\begin{frame}
%  \frametitle{What are Distributed Hash Tables?}
%  %\framesubtitle{Subtitles are optional.}
%  % - A title should summarize the slide in an understandable fashion
%  %   for anyone how does not follow everything on the slide itself.
%
%  \begin{itemize}
%    \item
%    What are they?
%    \begin{itemize}
%        \item
%            Decentralised
%        \item
%            Self-organising
%        \item
%            Fault-tolerant
%        \item
%            Overlay network
%    \end{itemize}
%    ~\\
%    \item
%    What do they do?
%    \begin{itemize}
%    \item
%        Efficient request routing
%    \item
%        Deterministic object location
%    \item
%        Load balancing
%    \end{itemize}
%    ~\\
%  \end{itemize}
%\end{frame}

%
%\pgfdeclareimage[height=7cm]{DHT1}{Diagrams/DHT1}
%\pgfdeclareimage[height=7cm]{DHT2}{Diagrams/DHT2}
%\pgfdeclareimage[height=7cm]{DHT3}{Diagrams/DHT3}
%\pgfdeclareimage[height=7cm]{DHT4}{Diagrams/DHT4}
%
%\begin{frame}
%  \frametitle{How do they work?}
%  \begin{center}
%  \pgfuseimage{DHT1}
%  \end{center}
%\end{frame}
%
%\begin{frame}
%  \frametitle{How do they work?}
%  \begin{center}
%  \pgfuseimage{DHT2}
%  \end{center}
%\end{frame}
%
%\begin{frame}
%  \frametitle{How do they work?}
%  \begin{center}
%  \pgfuseimage{DHT3}
%  \end{center}
%\end{frame}
%
%\begin{frame}
%  \frametitle{How do they work?}
%  \begin{center}
%  \pgfuseimage{DHT4}
%  \end{center}
%\end{frame}

\subsection[Motivations]{Motivations}

\begin{frame}
  \frametitle{Problems with DHTs and Mobility?}
    \begin{itemize}
    \item
        Assumes homogeneity\\
        \begin{itemize}
            \item All nodes are treated equally (routing, storing
            etc.)
            \item Similar bandwidth, processing power, uptime
            \item \alert{Mobile environments are very heterogenous!}
        \end{itemize}
%    \item
%        Churn (joins and leaves) generates overhead\\
%        \begin{itemize}
%            \item Join forces routing updates
%            \item Leaves make the routing tables stale
%            \item \alert{Mobile devices lose connectivity when out of range, changing Access
%            Point or when Battery dies}
%        \end{itemize}
    \item
        Security (or lack thereof)\\
        \begin{itemize}
            \item Sniffing, Man in the Middle, Routing Table Poisoning
            \item Difficulties in supporting user authentication
            \item \alert{Very easy to join/sniff wifi networks, need for increase
            security}
        \end{itemize}
    \item
        Churn
        \begin{itemize}
            \item Wait for next slide...
        \end{itemize}
%    \item
%        Digital Rights Management
%        \begin{itemize}
%            \item Unlicensed content may be put into the network
%        \end{itemize}
    \end{itemize}
\end{frame}

\begin{frame}
  \frametitle{Problems with DHTs and Mobility?}

    Mobility Churn\\
    \begin{itemize}
        \item Join forces routing updates
        \item Leaves make the routing tables stale
    \end{itemize}
    ~\\
    When does this happen?
    \begin{itemize}
        %\item \alert{Mobile devices lose connectivity when out of range, changing Access Point or when Battery dies}
        \item Loses connectivity when out of range
        \item Batteries prone to running dry
        \item When changing Access Point
            \begin{itemize}
                \item Hand-over time
                \item May retain IP address (No need to rejoin, but data may be stale)
                \item May change IP address (May need to rejoin, or re-announce)
            \end{itemize}
    \end{itemize}

\end{frame}

\subsection[Stealth DHTs - Overview]{Stealth DHTs - Overview}

\pgfdeclareimage[height=5.7cm]{Stealth}{Diagrams/Stealth}

%\subsubsection[What are they?]{What are they?}

\begin{frame}
  \frametitle{What are Stealth DHTs?}
  \begin{columns}

    \column{4.5cm}
    \pgfuseimage{Stealth}

    \column{6cm}
    \begin{itemize}
      \item Service Nodes
      \begin{itemize}
          \item Assumed to be the more capable nodes
          \item Responsible for forwarding and storing data
          \item e.g. \alert{Wired Node}
      \end{itemize}
      ~\\
      \item Stealth Nodes
      \begin{itemize}
          \item Assumed to be the less capable nodes
          \item Have no responsibilities
          \item Invisible to all nodes, including Service nodes
          \item e.g. \alert{Wireless Node}
      \end{itemize}
    \end{itemize}

  \end{columns}
\end{frame}

\begin{frame}
  \frametitle{How does it help?}
    \begin{itemize}
    \item
        \alert{Does not} assume homogeneity\\
        \begin{itemize}
            \item Nodes can now be divided based on their capabilities
            \item More powerful nodes, take more responsibility
        \end{itemize}
    \item
        Churn (joins and leaves) generates \alert{little to no} overhead\\
        \begin{itemize}
            \item Stealth join requires less overhead
            \item Joining of Stealth nodes does not require routing
            updates
            \item Stealth nodes leaving does not make routing tables stale
        \end{itemize}
    \item
        Security \alert{\sout{(or lack thereof)}}\\
        \begin{itemize}
            \item Authentication for the Service nodes ensure that
            only authorised nodes route and store message
            \item Stealth nodes cannot commit sniffing, corruption
            attacks
        \end{itemize}
%    \item
%        Digital Rights Management (\alert{well nearly})
%        \begin{itemize}
%            \item Only authenticated nodes are permitted to put
%            content into the network
%        \end{itemize}
    \end{itemize}
\end{frame}

\subsection[Stealth DHTs - How do they work?]{Stealth DHTs - How do they work?}

\pgfdeclareimage[height=6.5cm]{Join1}{Diagrams/Pastry1}
\pgfdeclareimage[height=6.5cm]{Join2}{Diagrams/Pastry2}
\pgfdeclareimage[height=6.5cm]{Join3}{Diagrams/Pastry3}
\pgfdeclareimage[height=6.5cm]{Join4}{Diagrams/Pastry4}
\pgfdeclareimage[height=6.5cm]{Join5}{Diagrams/Pastry5}
\pgfdeclareimage[height=6.5cm]{Join6}{Diagrams/Pastry6}
\pgfdeclareimage[height=6.5cm]{Join7}{Diagrams/Pastry7}
\pgfdeclareimage[height=6.5cm]{Join8}{Diagrams/Pastry8}

\begin{frame}
  \frametitle{Pastry's Join - State Gathering}
  \begin{center}
    \pgfuseimage{Join1}
  \end{center}
\end{frame}

\begin{frame}
  \frametitle{Pastry's Join - State Gathering}
  \begin{center}
  \pgfuseimage{Join2}
  \end{center}
\end{frame}

\begin{frame}
  \frametitle{Pastry's Join - State Gathering}
  \begin{center}
  \pgfuseimage{Join3}
  \end{center}
\end{frame}

\begin{frame}
  \frametitle{Pastry's Join - State Gathering}
  \begin{center}
  \pgfuseimage{Join4}
  \end{center}
\end{frame}

\begin{frame}
  \frametitle{Pastry's Join - State Gathering}
  \begin{center}
  \pgfuseimage{Join5}
  \end{center}
\end{frame}

\begin{frame}
  \frametitle{Pastry's Join - State Gathering}
  \begin{center}
  \pgfuseimage{Join6}
  \end{center}
\end{frame}

\begin{frame}
  \frametitle{Pastry's Join - State Gathering}
  \begin{center}
  \pgfuseimage{Join7}
  \end{center}
\end{frame}

\begin{frame}
  \frametitle{Pastry's Join - Announcement}
  \begin{center}
  \pgfuseimage{Join8}
  \end{center}
\end{frame}

\pgfdeclareimage[height=7cm]{SJoin1}{Diagrams/Join1}
\pgfdeclareimage[height=7cm]{SJoin2}{Diagrams/Join2}
\pgfdeclareimage[height=7cm]{SJoin3}{Diagrams/Join3}
\pgfdeclareimage[height=7cm]{SJoin4}{Diagrams/Join4}
\pgfdeclareimage[height=7cm]{SJoin5}{Diagrams/Join5}
\pgfdeclareimage[height=7cm]{SJoin6}{Diagrams/Join6}
\pgfdeclareimage[height=7cm]{SJoin7}{Diagrams/Join7}

\begin{frame}
  \frametitle{Stealth Node's Join - State Gathering}
  \begin{center}
    \pgfuseimage{SJoin1}
  \end{center}
\end{frame}

\begin{frame}
  \frametitle{Stealth Node's Join - State Gathering}
  \begin{center}
  \pgfuseimage{SJoin2}
  \end{center}
\end{frame}

\begin{frame}
  \frametitle{Stealth Node's Join - State Gathering}
  \begin{center}
  \pgfuseimage{SJoin3}
  \end{center}
\end{frame}

\begin{frame}
  \frametitle{Stealth Node's Join - State Gathering}
  \begin{center}
  \pgfuseimage{SJoin4}
  \end{center}
\end{frame}

\begin{frame}
  \frametitle{Stealth Node's Join - State Gathering}
  \begin{center}
  \pgfuseimage{SJoin5}
  \end{center}
\end{frame}

\begin{frame}
  \frametitle{Stealth Node's Join - State Gathering}
  \begin{center}
  \pgfuseimage{SJoin6}
  \end{center}
\end{frame}

\begin{frame}
  \frametitle{Stealth Node's Join - State Gathering}
  \begin{center}
  \pgfuseimage{SJoin7}
  \end{center}
\end{frame}

\begin{frame}
  \frametitle{Stealth Node's Join - Announcement}
  \begin{center}
  \alert{and NO announcement!}
  \end{center}
\end{frame}

\begin{frame}
  \frametitle{How does it work? - Summary}
  \begin{itemize}

    \item Service nodes join normally
    \item Stealth nodes join but do not announce
    \\~\\
    \item Therefore
      \begin{itemize}
        \item Stealth nodes never appear in any node's routing tables
        \item Stealth nodes still have complete routing tables, thus resistance and optimal routing (of their own messages)
        \item Stealth nodes are not responsible for routing, or storing keys, etc
        \item Stealth node's churn affects no one
        \item Stealth nodes are effectively invisible
      \end{itemize}
    \item However
      \begin{itemize}
        \item Stealth nodes won't receiving routing updates
      \end{itemize}

  \end{itemize}
\end{frame}

\begin{frame}
  \frametitle{How does it work? - Summary}
  However Stealth nodes don't receiving routing updates. (ie knowledge that new service nodes have joined)\\
  Therefore they have an increasingly stale routing table \\
  ~\\
  Three solutions to this:
  \begin{itemize}
    \item Polling for updates
    \item Piggy backing updates
    \item Periodically rejoining the network
  \end{itemize}
\end{frame}

\section{Evaluation}

\subsection[Methodology]{Methodology}

\begin{frame}
  \frametitle{Methodology}
  \begin{itemize}
  \item
    Implementation
    \begin{itemize}
      \item Wrote a Peer-to-Peer simulator in java
      \item Implemented both Pastry and Stealth DHT (based on
      Pastry)
    \end{itemize}
%    ~\\
  \item
    Constructed networks of 1-1000 peers
    \begin{itemize}
      \item 1000 Router transit-stub GT-ITM network (4\% transit nodes)
      \item Each stub/edge router was a wifi access point
      \item Connected Stealth nodes in a random fashion to the APs
      \item Connected Service nodes in a random fashion via wired
      links to the APs
%      \item Varying numbers of service and stealth nodes
    \end{itemize}
%    ~\\
  \item
    Simulations (Realistic Scenario)
    \begin{itemize}
      \item Place 1 million keys in the network
      \item Requested keys due to a Zipf distribution $\alpha = 1.2$
      \item With and Without Mobility Churn
        \begin{itemize}
            \item Random Waypoint Model with mean 60min "thinking" times
        \end{itemize}
    \end{itemize}
  \end{itemize}
\end{frame}

\pgfdeclareimage[height=6.5cm]{Mobile}{Diagrams/Mobile}
\begin{frame}
  \frametitle{Methodology}
  \begin{columns}

    \column{6cm}
    \pgfuseimage{Mobile}

    \column{5cm}
    \begin{itemize}
      \item Service Nodes
      \begin{itemize}
        \item PCs (workstation/servers etc)
        \item Connected via a wired Network
      \end{itemize}
      ~\\
      \item Stealth Nodes
      \begin{itemize}
        \item Mobile devices
        \item Connected via the wifi Access Points
      \end{itemize}
      ~\\
      \item Service/Stealth all in the same DHT
    \end{itemize}
  \end{columns}
\end{frame}

%\subsection[Results part 1]{Results part 1}
%
%    \begin{frame}
%  \frametitle{Average messages per node during join vs DHT size}
%  \begin{center}
%    \pgfdeclareimage[height=6.8cm]{join_per_peer}{Diagrams/join_per_peer}
%    \pgfuseimage{join_per_peer}
%  \end{center}
%\end{frame}
%
%\begin{frame}
%  \frametitle{Average DHT hops vs DHT size}
%  \begin{center}
%    \pgfdeclareimage[height=7cm]{hops}{Diagrams/hops}
%    \pgfuseimage{hops}
%  \end{center}
%\end{frame}

%\begin{frame}
%  \frametitle{Average DHT hops vs \# of Stealth nodes}
%  \begin{center}
%    \pgfdeclareimage[height=7cm]{hops_vs_stealth}{Diagrams/hops_vs_stealth}
%    \pgfuseimage{hops_vs_stealth}
%  \end{center}
%\end{frame}

%\begin{frame}
%  \frametitle{Stretch vs DHT size}
%  \begin{center}
%    \pgfdeclareimage[height=6cm]{churnstretch}{Diagrams/churnstretch}
%    \pgfuseimage{churnstretch}\\
%    Stretch = DHT End-to-end delay / Unicast End-to-end delay
%  \end{center}
%\end{frame}

%\begin{frame}
%  \frametitle{Recv'ed messages per node vs DHT size}
%  \begin{center}
%    \pgfdeclareimage[height=7cm]{recv_per_node}{Diagrams/recv_per_node}
%    \pgfuseimage{recv_per_node}\\
%  \end{center}
%\end{frame}
%
%\begin{frame}
%  \frametitle{Recv'ed messages per node vs DHT size (with Churn)}
%  \begin{center}
%    \pgfdeclareimage[height=7cm]{recv_per_node_churn}{Diagrams/recv_per_node_churn}
%    \pgfuseimage{recv_per_node_churn}\\
%  \end{center}
%\end{frame}

\subsection[Results]{Results}
%
%\begin{frame}
%  \frametitle{Network Stress vs DHT size}
%  \begin{center}
%    \pgfdeclareimage[height=7cm]{stress}{Diagrams/stress}
%    \pgfuseimage{stress}\\
%  \end{center}
%\end{frame}
%
%\begin{frame}
%  \frametitle{Network Stress vs DHT size (with Churn)}
%  \begin{center}
%    \pgfdeclareimage[height=7cm]{stress_churn}{Diagrams/stress_churn}
%    \pgfuseimage{stress_churn}\\
%  \end{center}
%\end{frame}

\begin{frame}
  \frametitle{Results - Introduction}
  \begin{itemize}
    \item All plots show 1000 Peer networks
    \begin{itemize}
        \item 1\% Service nodes
        \item 99\% Stealth nodes
    \end{itemize}
    ~\\
    \item Plots on the x-axis show fractions of Stealth nodes who were wireless vs wired
    ~\\~\\
    \item Moving \{Stealth, Pastry\} refers to simulations where wireless nodes
    moved from AP point to AP. (A new IP is obtained)
    \item Static \{Stealth, Pastry\} refer to simulations where nodes
    did not move
  \end{itemize}
\end{frame}

\begin{frame}
  \frametitle{Total number of messages}
  \begin{center}
    \pgfdeclareimage[height=6.75cm]{mobile_messages}{Diagrams/mobile_messagecount}
    \pgfuseimage{mobile_messages}
  \end{center}
\end{frame}

\begin{frame}
  \frametitle{Failed packets due to nodes being unreachable}

  \begin{center}
    \pgfdeclareimage[height=7cm]{mobile_unreachable}{Diagrams/mobile_unreachable}
    \pgfuseimage{mobile_unreachable}\\
  \end{center}

\end{frame}

\begin{frame}
  \frametitle{Average lookup latency}

  \begin{center}
    \pgfdeclareimage[height=7cm]{mobile_lookup}{Diagrams/mobile_lookup}
    \pgfuseimage{mobile_lookup}\\
  \end{center}

\end{frame}


\section{Summary}

\subsection[Summary and Outlook]{Summary and Outlook}

\begin{frame}
  \frametitle<presentation>{Summary}

  Stealth DHT
  \begin{itemize}
  \item
    \alert{Partitions the network} into two groups
  \item
    \alert{Increases DHT performance} in most areas
  \item
    \alert{Returns control} to the service operator
  \end{itemize}

  % The following outlook is optional.
  \vskip0pt plus.5fill
Outlook
  \begin{itemize}
    \item
      Investigate \alert{possible applications} to run on top of a Stealth DHT
      \begin{itemize}
        \item Content Distribution Networks
        \item Novel Peer-to-Peer Applications
      \end{itemize}

    \item
      Automatically decide who is Stealth/Service node, and
      change them on the fly

    \item
      Look in detail how this can be applied to MANETs
  \end{itemize}
\end{frame}

%\subsection[Current state of work]{Current state of work}
%
%\begin{frame}
%  \frametitle{Current state of work}
%
%  Stealth DHT
%  \begin{itemize}
%  \item
%    We have a C++ implementation running on planetlab
%  \item
%    We have an Authentication model to enforce roles (and improve security)
%  \end{itemize}
%
%\end{frame}

\subsection[Thank you]{Thank you}

\begin{frame}
  \frametitle{Thank you for listening}
  \begin{center}
    \alert{Questions?}
    \titlepage
  \end{center}
\end{frame}

\end{document}
