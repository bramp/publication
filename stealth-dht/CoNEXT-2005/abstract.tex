%\documentclass{acm_proc_article-sp}
\documentclass{sig-alt-release}
%\documentclass{sig-alt-release2}

\usepackage{subfigure}
\usepackage{epsfig}

\begin{document}

\conferenceinfo{CoNEXT'05,} {October 24--27, 2005, Toulouse,
France.}
\CopyrightYear{2005}
\crdata{1-59593-035-3/05/0010}

\title{
  Stealth Distributed Hash Table:\\
  Unleashing the Real Potential of Peer-to-Peer
}

\numberofauthors{1}

\author {
  \alignauthor Andrew Brampton, Andrew MacQuire, Idris A. Rai\\ Nicholas J. P. Race and Laurent Mathy\\
  \affaddr{Computing Department}\\
  \affaddr{Lancaster University}\\
  \email{\{brampton,macquire,rai,race,laurent\}@comp.lancs.ac.uk}
}

\maketitle

%\begin{abstract}
%\end{abstract}

% http://www.acm.org/class/1998/ccs98.txt
% A category with the (minimum) three required fields
%\category{H.4}{Information Systems Applications}{Miscellaneous}
%A category including the fourth, optional field follows...
%\category{D.2.8}{Software Engineering}{Metrics}[complexity measures, performance measures]

%\terms{Delphi theory}
%\keywords{ACM proceedings, \LaTeX, text tagging} % NOT required for Proceedings

\section{Introduction}
\label{sect:intro}

Distributed Hash Tables (DHTs) have been shown to be a promising
form of decentralised structured peer-to-peer networking, offering
substantial scalability and resilience. Unsurprisingly, there exist
numerous DHT systems~\cite{can01}\cite{pastry01}\cite{chord01}.
Primarily, DHTs serve as an object location service that can be used
as a substrate for multiple large-scale distributed applications
such as storage~\cite{past}\cite{oceanstore},
multicast~\cite{scribe}\cite{split} and load balancing
systems~\cite{kar04}\cite{bri05}.

Historically, the approach to DHT design has involved connecting an
overlay of {\em homogeneous} and {\em autonomous} nodes together.
The homogeneity arises from the fact that nodes are assumed to have
similar properties such as network bandwidth and network stability.
On the other hand, the autonomicity of nodes arises in the sense
that any node may join or leave the network, and perform any
operation supported by the DHT such as {\em routing messages} or
handling object references ({\em keys}) as they wish.

Realistically, measurement studies have shown that nodes are not
homogeneous~\cite{sgg02} and nodes are likely to be continually
joining and leaving the network in an unpredictable fashion, a
situation commonly referred to as {\em churn}. Many DHT systems will
simply break down under high churn~\cite{churn1}, unfortunately,
severe levels of churn are likely, especially for networks with
mobile nodes~\cite{mobilechurn}\cite{dhtmanet01}.

Another issue is that of trust among users: traditional
content-delivery models separate clients and servers, unlike
DHT-based content systems where the end-users play a role in the
content location and delivery. It is unwise to allow all nodes the
same privileges on the network due to the potential for
abuse~\cite{dhtsec}\cite{DDoS_p2p05}.

We propose  a {\em Stealth Distributed Hash Table} (Stealth DHT)
algorithm, which aims to preserve the advantages of DHT-based
systems, while offering greater security and control at the DHT
level. A Stealth DHT makes a subset of nodes effectively
``invisible'' to the routing decisions on the network. As a result,
invisible nodes will never receive any queries and therefore cannot
intercept nor reply to them.

\section{Stealth DHT}
\label{sect:stealth}

%% Does the commerical example need generalizing?
There are two types of nodes in a Stealth DHT, namely {\em service
nodes} and {\em stealth nodes}. Service nodes can execute all
operations supported by the DHT, while stealth nodes are prevented
from storing keys and forwarding data. In a commercial context, for
example, service nodes could be highly stable and capable machines
owned by a content service provider, and they are assumed to be
trusted; on the other hand, stealth nodes would be autonomous
devices owned by end-users who request service(s) from the provider.
Note however, that the assignment of role (service or stealth) to
nodes is application dependent and in no way prescribed or
constrained by the Stealth DHT itself.

To this end, the routing state of all nodes in the Stealth DHT
contains entries for only service nodes. Consequently, any node (and
in particular stealth nodes) can only send messages to service
nodes. The service nodes then forward the message (only via other
service nodes) to the destination, which will also be a service
node. In other words, stealth nodes cannot communicate with one
another, nor will service nodes communicate with stealth nodes for
any purpose other then replying directly to requests issued by
stealth nodes. Consequently, stealth nodes are invisible to each
other, and when ``quiet'', their presence is invisible to the
service nodes.

To achieve the differentiation between service and stealth nodes,
stealth nodes make use of a lightweight join mechanism. This
mechanism prevents the stealth node announcing their presence to the
network, thus keeping them out of the other nodes' routing tables.
Intuitively, when a stealth node joins, no routing updates are
required, and when a stealth node leaves, no routing entries become
stale.

As stealth nodes do not appear in the service nodes' routing tables,
the stealth nodes do not receive updates to their own tables. This
can lead to a stealth node having an increasingly stale routing
table. There are three possible low-cost mechanisms to tackle this,
piggybacking routing information on replies to stealth nodes,
periodically polling for routing state, and rejoining the node to
the network.

\section{Performance Evaluation}
\label{sect:eval}

We implemented both Pastry~\cite{pastry01} (i.e. a generic DHT) and
our Stealth DHT in our own discrete-event packet-level simulator.
The underlying network consisted of 1000 routers on a transit-stub
network (4\% transit nodes), generated with GT-ITM~\cite{gtitm}.
Peers were attached to the physical network in a random fashion.

Simulations were run with a realistic workload in which randomly
selected nodes performed {\em put} and {\em get} operations on a set
of 1,000,000 keys. These simulations were run for both Pastry and a
Stealth DHT, with and without churn in the network.

Regardless of churn Stealth DHTs outperform Pastry in many standard
DHT measurements, such as average hop count, relative delay penalty
(often referred to as stretch), join overhead and load balancing. We
also found that in a Stealth DHT increasing the number of stealth
nodes had no significant impact on these metrics.

Without churn, the cost of using a Stealth DHT was increased link
stress for a small percentage of the network, as well as higher load
for the service nodes. Under churn, however, generic DHTs generate a
large number of messages to detect and repair the failures. Overall,
under churn the Stealth DHT was found to have a lower average and
maximum link stress, as well as reduced average load per node. We
attribute this to the lightweight joining protocol for stealth
nodes, and the lack of repair messages required when stealth nodes
leave the DHT.

\section{Conclusion}
\label{sect:conclusion}

Most of the research in DHT-based systems has focused on performance
issues. However, a generic DHT design does not generally address
issues such as heterogeneous nodes in a network and security.

We proposed a new DHT paradigm called Stealth DHT that is capable of
tackling these problems. In particular, a Stealth DHT simplifies the
DHT operations for stealth nodes, which leads to improved
performance. In addition, by preventing stealth nodes from
forwarding messages, the service nodes can actually restrict the
content stored in the DHT and enforce control over network
operations, thus improving security.

Stealth DHT could support commercial applications where the
distributed and resilient aspect of the DHT is required, but where
the content stored in the network needs to be limited to licence
materials. Mobility is another area where Stealth DHTs could be
used, by restricting mobile devices to stealth nodes their churn
would not adversely affect the network.

Consequently, Stealth DHTs are poised to improve support for
existing DHT applications in real world communication environments
as well as enabling the commercial development of the technology.


\bibliographystyle{abbrv}
\begin{thebibliography}{10}

\bibitem{gtitm}
K.~Calvert and E.~Zegura.
\newblock Georgiatech internetwork topology models.
\newblock http://www.cc.gatech.edu/projects/gtitm.

\bibitem{split}
M.~Castro, P.~Druschel, A.-M. Kermarrec, A.~Nandi, A.~Rowstron, and
A.~Singh.
\newblock Splitstream: High-bandwidth multicast in a cooperative environment.
\newblock In {\em Proc. of ACM SOSP}, October 2003.

\bibitem{scribe}
M.~Castro, P.~Druschel, A.-M. Kermarrec, and A.~Rowstron.
\newblock {SCRIBE}: A large-scale and decentralised application-level multicast
  infrastructure.
\newblock {\em IEEE Journal on Selected Areas in Communications (JSAC) (Special
  issue on Network Support for Multicast Communications}, 20(8), 2002.

\bibitem{past}
P.~Druschel and A.~Rowstron.
\newblock A large-scale, persistent peer-to-peer storage utility.
\newblock In {\em Proceedings of the Eighth Workshop on Hot Topics in Operating
  Systems (HotOS VIII)}, pages 75--80, May 2001.

\bibitem{DDoS_p2p05}
D.~Dumitriu, E.~Knightly, A.~Kuzmanovic, I.~Stoica, and
W.~Zwaenepoel.
\newblock Denial-of-service resilience in peer-to-peer file sharing systems.
\newblock In {\em Proc. of ACM SIGMETRICS}, pages 38--49, June 2005.

\bibitem{bri05}
P.~B. Godfrey and I.~Stoica.
\newblock Heterogeneity and load balance in distributed hash tables.
\newblock In {\em Proc. of IEEE INFOCOM}, March 2005.

\bibitem{mobilechurn}
H.-C. Hsiao and C.-T. King.
\newblock Mobility churn in {DHT}s.
\newblock In {\em Proc. of the 1st International Workshop on Mobility in
  Peer-to-Peer Systems (MPPS'05) in conjunction with the 25th IEEE
  International Conference on Distributed Computing Systems (ICDCS'05)}, pages
  799--805, June 2005.

\bibitem{kar04}
D.~R. Karger and M.~Ruhl.
\newblock Simple efficient load balancing algorithms for peer-to-peer systems.
\newblock In {\em ACM Symposium on Parallelism in Algorithms and
  Architectures}, pages 36--43, June 2004.

\bibitem{oceanstore}
J.~Kubiatowicz.
\newblock Oceanstore: An architecture for global-scalable persistent storage.
\newblock In {\em Proc. of the ASPLOS 2000}, November 2000.

\bibitem{dhtmanet01}
H.~Pucha, S.~M. Das, and Y.~C. Hu.
\newblock How to implement {DHT}s in mobile ad hoc networks?
\newblock In {\em Proc. of the 10th ACM International Conference on Mobile
  Computing and Network (MobiCom 2004)}, September 2004.

\bibitem{can01}
S.~Ratnasamy, P.~Francis, M.~Handley, R.~Karp, and S.~Shenker.
\newblock A scalable content-addressable network.
\newblock In {\em Proc. of ACM SIGCOMM}, August 2001.

\bibitem{churn1}
S.~Rhea, D.~Geels, T.~Roscoe, and J.~Kubiatowicz.
\newblock Handling churn in a {DHT}.
\newblock In {\em Proc. of the USENIX Annual Technical Conference}, June 2004.

\bibitem{pastry01}
A.~Rowstron and P.~Druschel.
\newblock Pastry: Scalable, distributed object location and routing for
  large-scale peer-to-peer systems.
\newblock In {\em Proc. of the 18th IFIP/ACM International Conference on
  Distributed Systems Platforms}, November 2001.

\bibitem{sgg02}
S.~Saroiu, P.~K. Gummadi, and S.~D. Gribble.
\newblock A measurement study of peer-to-peer file sharing systems.
\newblock In {\em Proc. of MMCN}, 2002.

\bibitem{dhtsec}
E.~Sit and R.~Morris.
\newblock Security considerations for peer-to-peer distributed hash tables.
\newblock In {\em Proc. of IPTPS}, March 2002.

\bibitem{chord01}
I.~Stoica, R.~Morris, D.~Karger, F.~Kaashoek, and H.~Balakrishnan.
\newblock Chord: A scalable peer-to-peer lookup service for internet
  applications.
\newblock In {\em Proc. of ACM SIGCOMM}, pages 149--160, August 2001.

\end{thebibliography}
\end{document}
