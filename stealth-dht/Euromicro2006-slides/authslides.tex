\documentclass{beamer}


\mode<presentation> {
  \usetheme{Warsaw}
  \setbeamercovered{transparent}
}

\usepackage[english]{babel}

\usepackage[latin1]{inputenc}

\usepackage{times}
\usepackage[T1]{fontenc}

\usepackage{ulem}

\title[Stealth DHT]
{Authentication in\\Stealth Distributed Hash Tables}

\institute[Lancaster University, UK] {
  \textbf{Andrew MacQuire}, Andrew Brampton, Idris A. Rai, \\Nicholas J. P. Race and Laurent Mathy\\
  \{macquire,brampton,rai,race,laurent\}@comp.lancs.ac.uk\\
  Computing Department\\
  Lancaster University, UK
}

\date[30th August 2006]
{Wednesday 30th August \\ \textbf{Euromicro 2006}}

\subject{Stealth Distributed Hash Table}

 \pgfdeclareimage[height=1.1cm]{university-logo}{lancslogo}
 \logo{ \pgfuseimage{university-logo} }

\setbeamertemplate{navigation symbols}{}

\begin{document}

\begin{frame}
  \titlepage
\end{frame}

\begin{frame}
  \frametitle{Outline}
  \tableofcontents
\end{frame}

\section{Introduction}
\subsection{What is a Distributed Hash Table?}
\begin{frame}
  \frametitle{What is a Distributed Hash Table?}
  \begin{columns}

  \column{6cm}

  \begin{block}{Common Characteristics}
     \begin{itemize}
     \item{DHTs allow data sharing amongst \textbf{numerous} machines}
     \item{Nodes and data are \textbf{consistently identified} via hash functions}
     \item{Distributed routing algorithms allow \textbf{any node} to be located}
     \item{Peer-to-peer system: typically \textbf{all nodes are treated equally}}
     \end{itemize}
  \end{block}

  \column{4.5cm}

  \begin{center}
    \pgfdeclareimage[height=4.5cm]{dht}{diagrams/DHT}
    \pgfuseimage{dht}
  \end{center}

\end{columns}

\end{frame}

\subsection{What security problems exist?}
\begin{frame}
  \frametitle{What security problems exist?}
  \begin{columns}
  \column{4.5cm}

  \begin{center}
    \pgfdeclareimage[height=4.5cm]{through}{diagrams/Through}
    \pgfuseimage<1>{through}
    \pgfdeclareimage[height=4.5cm]{flood}{diagrams/Flood}
    \pgfuseimage<2>{flood}
  \end{center}

  \column{6cm}
  \begin{block}{Common Attacks}
    \begin{itemize}
    \item<1>{Messages are \textbf{routed through} intermediate nodes}
        \begin{itemize}
        \item{\it Sniffing}
        \item{\it Man-in-the-Middle}
        \item{\it Denial of Service}
        \end{itemize}
    \item<2>{Nodes can inject \textbf{unwanted} messages}
        \begin{itemize}
        \item{\it Spoofing}
        \item{\it Pollution}
        \end{itemize}
  \end{itemize}
  \end{block}
  \end{columns}
\end{frame}

\begin{frame}
  \frametitle{What can be done?}
  \begin{itemize}

  \item<1->{Could simply \textbf{deny access} to untrusted nodes...}
  \begin{itemize}
    \item{...might \textbf{exclude} well-behaved nodes unnecessarily}
  \end{itemize}
  ~\\ %added a bit of padding
  \item<2>{Better to \textbf{restrict} the operations untrusted nodes can perform}
  \begin{itemize}
    \item{...but how can nodes be \textbf{separated} into trusted and untrusted?}
  \end{itemize}
  \end{itemize}

\end{frame}

\section{Stealth Distributed Hash Tables}
\subsection{How can a Stealth DHT help?}
\begin{frame}
  \frametitle{How can a Stealth DHT help?}
  \begin{itemize}
  \item{Stealth DHTs separate nodes into \textbf{Service} and \textbf{Stealth} classes}
    \begin{itemize}
    \item{An altered join procedure (without phase `b') \textbf{stops} Stealth nodes from being routing \textbf{intermediaries}}
    \end{itemize}
  \end{itemize}

  \begin{center}
    \pgfdeclareimage[height=4.5cm]{join}{diagrams/Join}
    \pgfuseimage{join}
  \end{center}
\end{frame}

\subsection{How can nodes be authenticated?}
\begin{frame}
  \frametitle{How can nodes be authenticated?}

  \begin{itemize}
  \item{Service node status is given to \textbf{trusted} nodes only, all others become Stealth nodes}
    \begin{itemize}
    \item{...however, a system is required to \textbf{enforce} this separation}
    \end{itemize}
  \end{itemize}

  \begin{columns}



    \column{6cm}

  \begin{itemize}
  \item<2->{\textbf{Public Key Infrastructures}}
    \begin{itemize}
    \item{...allow multiple users to validate each other's identities and messages (as well as encrypt traffic) with \textbf{no prior exchange} of data}
    \end{itemize}
  \end{itemize}

  \column{5.0cm}
  \begin{block}<3->{Typical PKI Components}
    \begin{itemize}
    \item{\textbf{Registration Authority}}
    \item{\textbf{Certification Authority}}
    \item{\textbf{Certificate Repository}}
    \item{\it{Revocation List?}}
    \end{itemize}
  \end{block}

  \end{columns}

\end{frame}

\section{Authentication}
\subsection{How does a Stealth DHT PKI work?}
\begin{frame}
  \frametitle{How does a Stealth DHT PKI work?}
  \begin{itemize}
  \item{Could be a completely \textbf{external} system...}
  \item{Could also consist of one or more \textbf{internal} service nodes...}
    \begin{itemize}
    \item{The {\it Registration} and {\it Certification Authorities} \textbf{must be highly trusted}}
    \item{The {\it Certificate Repository} can be \textbf{any node(s)}, as certificates are immutable and digitally signed}
    \end{itemize}
  \end{itemize}
\end{frame}

\subsection{What implementation considerations exist?}
\begin{frame}
  \frametitle{What implementation considerations exist?}
  \begin{itemize}
  \item{\textbf{Certification Hierarchy}}
    ~\\
    \begin{itemize}
    \item{\it Single globally trusted key?}
       \begin{itemize}
       \item{\textbf{Simple}, but problematic if the key is ever \textbf{compromised}}
       \end{itemize}
    ~\\
    \item{\it Hierarchy of keys?}
       \begin{itemize}
       \item{More \textbf{complex}, but allows for finer-grained \textbf{control}}
       \item{\textbf{Highest-level} keys can be kept secure and used \textbf{rarely}}
       \end{itemize}
    \end{itemize}
  \end{itemize}
\end{frame}

\begin{frame}
  \frametitle{What implementation considerations exist?}
  \begin{itemize}
  \item{\textbf{Permissions Management}}
    \begin{itemize}
    \item{\it Permissions within certificates?}
    ~\\
       \begin{itemize}
       \item{\textbf{Simple}, but \textbf{increases} message size}
       \item{Does \textbf{not} require an extra lookup}
       \item{Difficult to alter; certificates are \textbf{immutable}}
       \end{itemize}
    ~\\
    \item{\it Hold permissions on dedicated service node(s)?}
       \begin{itemize}
       \item{\textbf{Complex}, yet \textbf{flexible}}
       \item{\textbf{Increases} messaging overhead through lookups}
       \end{itemize}
    \end{itemize}
  \end{itemize}
\end{frame}

\begin{frame}
  \frametitle{What implementation considerations exist?}
  \begin{itemize}
  \item{\textbf{Certificate Revocation}}
    \begin{itemize}
    \item{\it Short expiry times?}
    ~\\
       \begin{itemize}
       \item{Nodes are \textbf{re-issued} certificates periodically; unwanted certificates are \textbf{not} re-issued}
       \item{\textbf{Tradeoff} between certification overhead and response time}
       \end{itemize}
    ~\\
    \item{\it Certificate Revocation List?}
       \begin{itemize}
       \item{List stored/replicated across dedicated service node(s)}
       \item{Polled periodically or per-authentication operation; \textbf{another tradeoff} between overhead and response time}
       \end{itemize}
    \end{itemize}
  \end{itemize}
\end{frame}

\begin{frame}
  \frametitle{What implementation considerations exist?}
  \begin{itemize}
  \item{\textbf{Authentication Granularity}}
    \begin{itemize}
    \item{\it How often are messages validated?}
    ~\\
       \begin{itemize}
       \item{Per-\textbf{hop}?}
       \item{Per-\textbf{message}?}
       \item{Per-\textbf{session}?}
       \end{itemize}
    ~\\
    \item{\it Are complete certificates chains included in messages?}
       \begin{itemize}
       \item{\textbf{Increased} message size, but \textbf{no extra messaging} required}
       \end{itemize}
    \end{itemize}
  \end{itemize}
\end{frame}

\begin{frame}
  \frametitle{Percentage increase in number of messages...}

  \begin{center}
    \pgfdeclareimage[height=6cm]{granularity}{diagrams/Granularity}
    \pgfuseimage{granularity}
  \end{center}

\end{frame}

\begin{frame}
  \frametitle{Percentage increase in lookup latency...}

  \begin{center}
    \pgfdeclareimage[height=6cm]{granularity2}{diagrams/Granularity2}
    \pgfuseimage{granularity2}
  \end{center}

\end{frame}

\section{Conclusion}
\begin{frame}
  \frametitle{Conclusion}
  \begin{itemize}
  \item{Stealth DHTs were originally proposed to separate \textbf{powerful from less powerful} nodes}
  \item{This distinction can be extended to \textbf{trusted and untrusted} nodes}
  \item{By \textbf{limiting} the operations stealth nodes can perform, untrusted nodes \textbf{can still be serviced}}
  \item{By \textbf{enforcing} the separation through a \textbf{PKI}, Stealth DHTs can supply a \textbf{secure, resilient overlay}}
  \end{itemize}
\end{frame}

\begin{frame}
  \frametitle{Thank you for listening.}
  \begin{center}
    \textbf{Any questions?}
    \titlepage
  \end{center}
\end{frame}

\end{document}
